%%This is a very basic article template.
%%There is just one section and two subsections.
\documentclass{article}

\usepackage{appendix}
\usepackage{amsmath}
\usepackage{caption}
\usepackage{placeins}
\usepackage{graphicx}
\usepackage{subcaption}
%\usepackage{subfig}
\usepackage{longtable}
\usepackage{setspace}
%\usepackage{tikz}
\usepackage{booktabs}
\usepackage{tabularx}
\usepackage{xcolor,colortbl}
\usepackage{chngpage}
%\usepackage[active,tightpage]{preview}
\usepackage{natbib}
\bibpunct{(}{)}{,}{a}{}{;} 
\usepackage{url}
\usepackage{nth}
\usepackage{authblk}
\usepackage[most]{tcolorbox}
\usepackage{xcolor}
\usepackage{array}
\newcolumntype{C}{ >{\centering\arraybackslash} m{5.5cm} }
\newcolumntype{D}{ >{\centering\arraybackslash} m{1cm} }
%\usepackage{hyperref}
%\usepackage{color}
%\usepackage{fontspec}
%\usepackage{pdfsync}
\usepackage[normalem]{ulem}
\usepackage{amsfonts}
\usepackage{xfrac}
% for the d in integrals
\newcommand{\dd}{\; \mathrm{d}}
\newcommand{\tc}{\quad\quad\text{,}}
\newcommand{\tp}{\quad\quad\text{.}}
\newcommand{\scen}[1]{\includegraphics[scale=.5]{Figures/scenarios/{scenario#1}.pdf}}
\defcitealias{HMD}{HMD}
\defcitealias{HFD}{HFD}

\begin{document}

\title{Morbidity concentration and compression}

\author[1]{Tim Riffe\thanks{riffe@demogr.mpg.de}}
\author[2]{A{\"i}da Sol\'{e} i Auro}
\author[1]{Maarten Bijslma}
\affil[1]{Max Planck Institute for Demographic Research}
\affil[2]{Universitat Pompeu Fabra}

\maketitle

The concept of morbidity compression is typically instrumentalized by taking a
ratio of expected years lived in poor health, $e^U$ to total expected years
lived, $e^T$.
It is therefore a proportion derived from two means, and it is usually conditioned on survival to some age. If after some time this proportion
decreases, then morbidity is said to have compressed. This is intuitive because on average a smaller proportion of life is spent in
poor health, and therefore morbidity has been squeezed into a relatively smaller
fraction of life.

Various changes in morbidity patterns and longevity
distributions can produce a change in compression, but this single
measure cannot separate them. Specifically, and for the case of late-life morbidity, the measure of
morbidity compression does not indicate to what extent morbidity prevalence is
concentrated in the final years of life. All else equal, we suppose that
individual both wellbeing and equality are maximized for a given level of
morbidity and longevity if morbidity prevalence is concentrated in a narrow set of years at the very end of
life.
We therefore propose a direct measure of late-life morbidity concentration that
is both intuitive and clear to interpret: how close to to the moment of death a given morbidity is
concentrated, calculated as the prevalence-weighted average
time-to-death, and we call this measure $\Phi$. 

\section*{Definitions}
Compression, $\kappa$, is defined as:
\begin{align}
\kappa(a) &= \frac{e^U(a)}{e^T(a)}
\intertext{where}
e^T(a) &=\frac{1}{\ell(a)} \int_a^\omega \ell(x) \dd x \\
e^U(a) &=\frac{1}{\ell(a)} \int_a^\omega \ell(x)\pi(x) \dd x
\end{align}
Here, $\kappa$ is conditional on survival to age $a$, because we are
interested in old-age morbidity, so we typically choose $a$ to be something like
65 or 70. $l(x)$ is lifetable survivorship, and $\pi(x)$ is the
age-specific prevalence pattern of morbidity.

Concentration, $\Phi$, on the other hand measures the compactness of morbidity
at the end of life. Assuming for a moment perfect equality of longevity, it
can be defined as:
\begin{equation}
\Phi = \frac{\int_a^\omega (\omega - x)\pi(x) \dd x}{\int_a^\omega \pi(x) \dd
x}
\end{equation}
Slightly more realistically, since lives are of different lengths, we can
still keep things simple by letting longevity vary, but assuming the same
time-to-death trajectory of morbidity, $\pi^\star(y)$, where $y$ denotes years
left to live:
\begin{equation}
\Phi = \frac{\int_0^{\omega-a} y \pi^\star(y) \dd
y}{\int_0^{\omega-a}\pi^\star(y)\dd y}
\end{equation}
If we want to allow both lifespans and the time-to-death trajectory to vary,
then define a morbidity prevalence function that varies by both age and
time-to-death such as those described by \citet{riffe2015ttd}.

\textit{add equation here}
%Compression is a meaningful measure, but we aim to enrich it by deriving a
%synthetic measure of morbidity concentration. The index of morbidity
%concentration is interpreted as how much a given morbidity condition is
%concentrated in the final year(s) of life. This index is based on a more
%nuanced breakdown of morbidity prevalence by age and age-at-death (lifespan). 
%

\section*{Morbidity and longevity scenarios}
Several morbidity and longevity scenarios can produce a given change in 
compression. For example, morbidity may change its level, onset may
postpone, or lengthening life may stretch morbidity into ever
higher ages. These aspects of morbidity are all illustrated in the morbidity
diagrams of James Fries \citep[e.g.,][]{fries2003measuring}, but such
patterns are not directly detectable or separable using the measure of morbidity
compression. Another simple scenario is that the morbidity prevalence leading up to each age at death remains the
same, but the lifespan distribution simply changes \citep{vanRaalte2015HLE}.
Such aspects characterize late-life morbidity more completely, and provide
better answers to the question of whether there is a survival-morbidity tradeoff.

Table~\ref{tab:scenarios} illustrates a set of Fries-like scenarios that is
instructive to walk through. Assume that for a given scenario it is either the
case that all lives are of the same length, or that lives of different lengths
all have the same average end-of-life morbidity trajectory. The length of
lifeline in each scenario is proportional to a conditional remaining lifespan,
while the area shaded in grey is average morbidity prevalence. Scenarios A
through E all have a length of life equal to the baseline scenario, while F
through H have longer lives. Scenarios A, B, F, and H have morbidity triangles
of equal area to the baseline (equal $e^U$). Scenarios E and G have morbidity
triangles that are larger than baseline (higher $e^U$). Scenarios C and D have
morbidity triangles drawn smaller than baseline (lower $e^U$). Scenarios A, D,
F, and H have postponed morbidity onset.

 \begin{table}[h!]
\begin{tabular}{D C D D D D}
~ & Scenario & $e^U$ & $e^T$ & $\kappa$ & $\Phi$ \\
\hline
Base & \scen{1} & ~ & ~ & ~ & ~ \\
A & \scen{2} & $=$ & $=$ & $=$ & $\downarrow$ \\
B & \scen{8} & $=$ & $=$ & $=$ & $\uparrow$ \\
C & \scen{3} & $\downarrow$ & $=$ & $\downarrow$ &  $=$ \\
D & \scen{9} & $\downarrow$ & $=$ & $\downarrow$ &  $\downarrow$ \\
E & \scen{4} & $\uparrow$ & $=$ & $\uparrow$ &  $=$ \\
F & \scen{5} & $=$ & $\uparrow$ & $\downarrow$ &  $=$ \\
G & \scen{6} & $\uparrow$ & $\uparrow$ & $\uparrow$ &  $\uparrow$ \\
H & \scen{7} & $=$ & $\uparrow$ & $\downarrow$ &  $\downarrow$ \\
\end{tabular}
\caption{A variety of morbidity and longevity scenarios that illustrate how
compression differs from concentration.}
\label{tab:scenarios}
\end{table}

The arrows in table~\ref{tab:scenarios} for compression, $\kappa$ and
concentration $\Phi$ may be counterintuitive at first, because they refer to the
value of the measure. High compression and concentration are close to zero.
Scenarios A through H combine in different ways to produce different outcomes
for $\kappa$ and $\Phi$, and in many cases these to measures do not change in
the same direction. Clearly the best scenario here is for baseline to change to
H, while the worst scenarios are either E or G (higher morbidity prevalence
with no increase in lifespan, or else increases in both).

\textit{add more description}

\section*{Data and methods}
say we'll use HRS and ELSA, and natural splines in a general additive framework.
\section*{Results}
\textit{no need to show anything yet}
\section*{Conclusions}
Compression is a meaningful measure, but we aim to enrich it by deriving a
synthetic measure of morbidity concentration. The index of morbidity
concentration is interpreted as how much a given morbidity condition is
concentrated in the final year(s) of life. This index is based on a more
nuanced breakdown of morbidity prevalence by age and age-at-death (lifespan). 

\bibliographystyle{plainnat}
  \bibliography{references.bib}  

\end{document}

