%%This is a very basic article template.
%%There is just one section and two subsections.
\documentclass{article}

\usepackage{amsmath}
\usepackage{caption}
\usepackage{placeins}
\usepackage{graphicx}
\usepackage{subcaption}
\usepackage{tikz}
%\usepackage{mathabx} % for \widebar
%\usepackage[active,tightpage]{preview}
\usepackage{natbib}
\bibpunct{(}{)}{,}{a}{}{;} 
\usepackage{url}
\usepackage{nth}
\usepackage{authblk}
\usepackage{xfrac}
% for the d in integrals
\newcommand{\dd}{\; \mathrm{d}}
\newcommand{\tc}{\quad\quad\text{,}}
\newcommand{\tp}{\quad\quad\text{.}}
\defcitealias{HMD}{HMD}
\defcitealias{HFD}{HFD}
\usepackage[titletoc,title]{appendix}


\begin{document}


\title{Morbidity concentration and compression}

\author[1]{Tim Riffe\thanks{riffe@demogr.mpg.de}}
\author[2]{A{\"i}da Sol\'{e} Aur\'{o}}
\affil[1]{Max Planck Institute for Demographic Research}
\affil[2]{Universitat Pompeu Fabra}

\maketitle

The concept of morbidity compression is typically instrumentalized by taking a
ratio of expected years lived in poor health, $e^U$, to total expected years lived,
$e^T$, $\sfrac{e^U}{e^T}$. It is therefore a proportion derived from two means,
usually conditioned on survival to some age. If after some time this proportion
decreases, then morbidity is said to have compressed. This is intuitive because on average a smaller proportion of the average years lived is spent in poor
health. Morbidity has been squeezed into a relatively smaller section of
life. However, not much else can be said from this measure:
it says nothing of how morbidity is spread over ages, how much of the change in compression is
due to changes in the shape of morbidity, how much of it is due to postponement in
morbidity onset, or whether the morbidity preceding each age at death remained
the same, but the lifespan distribution simply changed. Such overlooked measures
characterize late-life morbidity more completely, and provide better answers to
the question of whether there is a survival-morbidity tradeoff. 

Compression is a meaningful measure, but we aim to enrich it by deriving a
synthetic measure of morbidity concentration based on a more nuanced breakdown
of morbidity prevalence by age and age-at-death (lifespan). 




\end{document}

\bibliographystyle{plainnat}
  \bibliography{references}  
