%%This is a very basic article template.
%%There is just one section and two subsections.
\documentclass{article}
\usepackage{natbib}
\usepackage{amsmath}
\usepackage{caption}
\usepackage{placeins}
\usepackage{graphicx}
\usepackage{subcaption}
\usepackage{tikz}
%\usepackage{mathabx} % for \widebar
%\usepackage[active,tightpage]{preview}
\usepackage{natbib}
\bibpunct{(}{)}{,}{a}{}{;} 
\usepackage{url}
\usepackage{nth}
\usepackage{authblk}
\usepackage{xfrac}
% for the d in integrals
\newcommand{\dd}{\; \mathrm{d}}
\newcommand{\tc}{\quad\quad\text{,}}
\newcommand{\tp}{\quad\quad\text{.}}
\defcitealias{HMD}{HMD}
\defcitealias{HFD}{HFD}
\usepackage[titletoc,title]{appendix}


\begin{document}

\title{Morbidity concentration and compression}

\author[1]{Tim Riffe\thanks{riffe@demogr.mpg.de}}
\author[2]{A{\"i}da Sol\'{e} i Aur\'{o}}
\affil[1]{Max Planck Institute for Demographic Research}
\affil[2]{Universitat Pompeu Fabra}

\maketitle

The concept of morbidity compression is typically instrumentalized by taking a
ratio of expected years lived in poor health, $e^U$, to total expected years lived,
$e^T$, $\sfrac{e^U}{e^T}$. It is therefore a proportion derived from two means,
and it is usually conditioned on survival to some age. If after some time this
proportion decreases, then morbidity is said to have compressed. This is
intuitive because on average a smaller proportion of life expectancy is spent in
poor health, and therefore morbidity has been squeezed into a relatively smaller
fraction of life. Various changes in morbidity patterns and longevity
distributions can produce a change in compression, but this single
measure can not separate them. Specifically, and for the case of late-life morbidity, the measure of
morbidity compression does not indicate to what extent morbidity prevalence is
concentrated in the final years of life. All else equal, we suppose that
individual wellbeing is maximized for a given level of morbidity if
morbidity prevalence is concentrated at the very end of life. We therefore
propose a direct measure of late-life morbidity concentration that is both intuitive and clear to interpret: the prevalence-weighted average time-to-death.

Compression is a meaningful measure, but we aim to enrich it by deriving a
synthetic measure of morbidity concentration. The index of morbidity
concentration is interpreted as how much a given morbidity condition is
concentrated in the final year(s) of life. This index is based on a more
nuanced breakdown of morbidity prevalence by age and age-at-death (lifespan). 


Several morbidity and longevity scenarios can produce a given change in 
compression. For example, morbidity may change its level, onset may
postpone, or lengthening life may stretch morbidity into ever
higher ages. These are all aspects of morbidity illustrated in the morbidity
diagrams of James Fries \citep[e.g.,][]{fries2003measuring}, but not directly detectable
or separable using the measure of morbidity compression. Another simple scenario
is that the morbidity prevalence leading up to each age at death remains the
same, but the lifespan distribution simply changes. Finally,   Such overlooked measures
characterize late-life morbidity more completely, and provide better answers to the question of whether there is a survival-morbidity tradeoff.

Compression is a meaningful measure, but we aim to enrich it by deriving a
synthetic measure of morbidity concentration. The index of morbidity
concentration is interpreted as how much a given morbidity condition is
concentrated in the final year(s) of life. This index is based on a more
nuanced breakdown of morbidity prevalence by age and age-at-death (lifespan). 




\end{document}

\bibliographystyle{plainnat}
  \bibliography{references.bib}  
