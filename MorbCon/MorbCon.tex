%%This is a very basic article template.
%%There is just one section and two subsections.
\documentclass{article}
\usepackage[top=2.5cm, bottom=2.5cm, left=3cm, right=3cm]{geometry}

\usepackage{appendix}
\usepackage{amsmath}
\usepackage{amsfonts}
\usepackage{caption}
\usepackage{placeins}
\usepackage{graphicx}
\usepackage{subcaption}
%\usepackage{subfig}
\usepackage{longtable}
\usepackage{setspace}
%\usepackage{tikz}
\usepackage{booktabs}
\usepackage{tabularx}
\usepackage{xcolor,colortbl}
\usepackage{chngpage}
%\usepackage[active,tightpage]{preview}
\usepackage{natbib}
\bibpunct{(}{)}{,}{a}{}{;} 
\usepackage{url}
\usepackage{nth}
\usepackage{authblk}
\usepackage[most]{tcolorbox}
\usepackage{xcolor}
\usepackage{array}
\newcolumntype{C}{ >{\centering\arraybackslash} m{5.5cm} }
\newcolumntype{D}{ >{\centering\arraybackslash} m{1cm} }
%\usepackage{hyperref}
%\usepackage{color}
%\usepackage{fontspec}
%\usepackage{pdfsync}
\usepackage[normalem]{ulem}
\usepackage{amsfonts}
\usepackage{xfrac}
% for the d in integrals
\newcommand{\dd}{\; \mathrm{d}}
\newcommand{\tc}{\quad\quad\text{,}}
\newcommand{\tp}{\quad\quad\text{.}}
\newcommand{\scen}[1]{\includegraphics[scale=.5]{Figures/scenarios/{scenario#1}.pdf}}
\defcitealias{HMD}{HMD}
\defcitealias{HFD}{HFD}

\begin{document}

\title{Morbidity dispersion and compression}

\author[1]{Tim Riffe\thanks{riffe@demogr.mpg.de}}
\author[2]{A{\"i}da Sol\'{e} Auro}
\author[1]{Maarten J. Bijlsma}
\affil[1]{Max Planck Institute for Demographic Research}
\affil[2]{Universitat Pompeu Fabra}

\maketitle

\begin{abstract}
The common measure of morbidity compression is insensitive to a variety of
shape-related aspects of late-life morbidity prevalence. We offer a simple
dispersion measure to account for how much a given morbidity condition is
concentrated in at the very end of life. We estimate morbidity compression and
dispersion for a set of cohorts in the US Health and Retirement Study and the
English Longitudinal Study of Ageing. We argue that all else equal, it is preferable for morbidity to be concentrated rather than dispersed, and this is why the measure is worth calculating alongside compression when data permits.
\end{abstract}

The concept of morbidity compression is typically instrumentalized by taking a
ratio of expected years lived in poor health, $e^U$ to total expected years
lived, $e^T$.
It is therefore a proportion derived from two means, and it is usually conditioned on survival to some age. If after some time this proportion
decreases, then morbidity is said to have compressed. This is intuitive because on average a smaller proportion of life is spent in
poor health, and therefore morbidity has been squeezed into a relatively smaller
fraction of life.

Various changes in morbidity patterns and longevity
distributions can produce a change in compression, but this single
measure cannot separate them. Specifically, and for the case of late-life morbidity, the measure of
morbidity compression does not indicate to what extent morbidity prevalence is
concentrated in the final years of life. All else equal, we suppose that
individual wellbeing and equality are both maximized for a given level of
morbidity and longevity if morbidity prevalence is concentrated in a narrow set of years at the very end of
life.
We therefore propose a direct measure of late-life morbidity dispersion that
is both intuitive and clear to interpret: how close to to the moment of death a given morbidity is
concentrated, calculated as the prevalence-weighted average
time-to-death, or dispersion, $\mathbb{D}$.

\section*{Definitions}
Compression, $\mathbb{C}$, is here defined as:
\begin{align}
\mathbb{C}(a) &= 1 - \frac{e^U(a)}{e^T(a)} \tc
\intertext{where}
e^T(a) &=\frac{1}{\ell(a)} \int_a^\omega \ell(x) \dd x \\
e^U(a) &=\frac{1}{\ell(a)} \int_a^\omega \ell(x)\pi(x) \dd x \tp
\end{align}
Here, $\mathbb{C}$ is conditional on survival to age $a$, because we are
interested in old-age morbidity, so we typically choose $a$ to be something like
65 or 70. $l(x)$ is lifetable survivorship, and $\pi(x)$ is the
age-specific prevalence of morbidity. We take the complement of the
ratio so that \textit{more} compressed corresponds to a higher value in
$\mathbb{C}$.

$\mathbb{D}$, on the other hand measures the spread of morbidity
at the end of life. Assuming for a moment perfect equality of longevity, it
can be defined as:
\begin{equation}
\mathbb{D} = \frac{\int_a^\omega (\omega - x)\pi(x) \dd x}{\int_a^\omega \pi(x) \dd
x} \tp
\end{equation}
Slightly more realistically, since lives are of different lengths, we can
still keep things simple by letting longevity vary, but assuming the same
time-to-death trajectory of morbidity, $\pi^\star(y)$, where $y$ denotes years
left to live:
\begin{equation}
\mathbb{D} = \frac{\int_0^{\omega-a} y \pi^\star(y) \dd
y}{\int_0^{\omega-a}\pi^\star(y)\dd y}
\end{equation}
If we want to allow both lifespans and the time-to-death trajectories to vary
by lifespan, then define a morbidity prevalence function that varies by both age
and time-to-death, $\pi^\star(a,y)$ such as the cohort patterns described by
\citet{riffe2015ttd}. In this case prevalence is weighted by the deaths
distribution, usually the lifetable deaths distribution, $f(a)$. In this case we
assume a unity radix, and we must decide between period and cohort perspectives.
We will elaborate on this in the full paper.

%\textit{add equation here}
%Compression is a meaningful measure, but we aim to enrich it by deriving a
%synthetic measure of morbidity concentration. The index of morbidity
%concentration is interpreted as how much a given morbidity condition is
%concentrated in the final year(s) of life. This index is based on a more
%nuanced breakdown of morbidity prevalence by age and age-at-death (lifespan). 
%

\section*{Morbidity and longevity scenarios}
Several morbidity and longevity scenarios can produce a given change in 
dispersion. For example, morbidity may change its level, onset may
postpone, or lengthening life may stretch morbidity into ever
higher ages. These aspects of morbidity are all illustrated in the morbidity
diagrams of James Fries \citep[e.g.,][]{fries2003measuring}, but such
patterns are not directly detectable or separable using the measure of morbidity
compression. Another simple scenario is that the morbidity prevalence leading up to each age at death remains the
same, but the lifespan distribution simply changes \citep{vanRaalte2015HLE}.
Such aspects characterize late-life morbidity more completely, and provide
better answers to the question of whether there is a survival-morbidity tradeoff.

Table~\ref{tab:scenarios} illustrates a set of Fries-like scenarios that is
instructive to walk through. Assume that for a given scenario it is either the
case that all lives are of the same length, or that lives of different lengths
all have the same average end-of-life morbidity prevlence pattern. The length of
the lifeline in each scenario is proportional to (conditional) remaining
lifespan, while the area shaded in grey is the average morbidity prevalence
pattern.
Scenarios A through E all have a length of life equal to the baseline scenario, while F through H have longer lives. Scenarios A, B, F, and H have morbidity triangles
of equal area to the baseline (equal $e^U$). Scenarios E and G have morbidity
triangles that are larger than baseline (higher $e^U$). Scenarios C and D have
morbidity triangles drawn smaller than baseline (lower $e^U$). Scenarios A, D,
F, and H have postponed morbidity onset.

 \begin{table}[ht!]
\begin{tabular}{D C D D D D}
~ & Scenario & $e^U$ & $e^T$ & $\mathbb{C}$ & $\mathbb{D}$ \\
\hline
Base & \scen{1} & ~ & ~ & ~ & ~ \\
A & \scen{2} & $=$ & $=$ & $=$ & $\downarrow$ \\
B & \scen{8} & $=$ & $=$ & $=$ & $\uparrow$ \\
C & \scen{3} & $\downarrow$ & $=$ & $\downarrow$ &  $=$ \\
D & \scen{9} & $\downarrow$ & $=$ & $\downarrow$ &  $\downarrow$ \\
E & \scen{4} & $\uparrow$ & $=$ & $\uparrow$ &  $=$ \\
F & \scen{5} & $=$ & $\uparrow$ & $\downarrow$ &  $=$ \\
G & \scen{6} & $\uparrow$ & $\uparrow$ & $\uparrow$ &  $\uparrow$ \\
H & \scen{7} & $=$ & $\uparrow$ & $\downarrow$ &  $\downarrow$ \\
\end{tabular}
\caption{A variety of morbidity and longevity scenarios that illustrate how
compression differs from concentration.}
\label{tab:scenarios}
\end{table}

Scenarios A through H combine in different ways to produce different outcomes
for $\mathbb{C}$ and $\mathbb{D}$, and in many cases these two measures do not
change in the same direction. Clearly the best scenario here is for baseline to change to
H, while the worst scenarios are either E or G (higher morbidity prevalence
with no increase in lifespan, or else increases in both). We now discuss the
scenarios in order.

Scenario A yields the same morbidity compression, but lower diserpersion.
Morbidity amounts to the same total, but is more concetrated at the end of life.
Scenario B is the opposite: compression does not change, but morbidity extends
into lower ages and ultimately \textit{may} affect fewer individuals. In
scenario C each value of $\pi(a)$ is halved, increasing
$\mathbb{C}$. However, the underlying shape does not change, and $\mathbb{D}$
therefore stays the same. In scenario D morbidity onset is later, but the
ultimate prevalence at death is the same. The volume decreases, and $\mathbb{C}$
therefore increases, but the shape also becomes more compact, which decreases
$\mathbb{D}$. In scenario E $\pi(a)$ is doubled, thereby decreasing $\mathbb{C}$.
Since the underlying shape does not change, $\mathbb{D}$ remains unchanged. In
scenerio F morbidity is exactly the same, but onset is postponed, and life
extended an equal amount. This acts to decrease $\mathbb{C}$, but it does not
affect $\mathbb{D}$. In scenario G life extends, and the prevalence pattern
keeps increasing at the same pace from the baseline scenario (Fries' stretching
scenario), thereby increasing both $\mathbb{C}$ and $\mathbb{D}$. H is the
golden scenario where increasing longevity is matched with a more compact and
less voluminous morbidity pattern, which acts to decrease both $\mathbb{C}$ and
$\mathbb{D}$.

We have provided scenarios in which compression and dispersion respond
in the same way or differently to various longevity and morbidity scenarios. We
have not shown how these different scenarios would produce different
responses in $\mathbb{C}$ and $\mathbb{D}$ under changes in the
shape or location of the lifespan distribution, nor more involved
interaction scenarios. These things may be discussed in the final paper, or we
might just make some assumptions to keep it simple.
\FloatBarrier
\section*{Data and methods}
We will do some empirical demonstrations to compare these two indicators for a
small set of health conditions that is to be decided. The health conditions will
be selected based on both substantive and illustrative criteria. Data will come
from the RAND version of the US Health and Retirement Study \citep{HRS} and
the English Longitudinal Study of Ageing \citep{steptoe2012cohort}. Data
processing will be set up similarly to \citet{riffe2015ttd}, and smoothed
prevalences by age and time-to-death will be derived using natural splines
over age, time-to-death, and birth cohort, in a general additive framework. We
will then report trends in compression versus dispersion for England and the US,
and further explain why dispersion is a good measure to complement compression.

\bibliographystyle{plainnat}
  \bibliography{references.bib}  

\end{document}

